\documentclass[12pt]{article}

\usepackage{graphics}
\usepackage{epsfig}
\usepackage{times}
\usepackage{amsmath}
%\usepackage{natbib}
%\usepackage[style=authoryear]{biblatex}
% \addbibresource{mendely.bib} 
% <http://psl.cs.columbia.edu/phdczar/proposal.html>:
%
% The standard departmental thesis proposal format is the following:
%        30 pages
%        12 point type
%        1 inch margins all around = 6.5   inch column
%        (Total:  30 * 6.5   = 195 page-inches)
%
% For letter-size paper: 8.5 in x 11 in
% Latex Origin is 1''/1'', so measurements are relative to this.

\topmargin      0.0in
\headheight     0.0in
\headsep        0.0in
\oddsidemargin  0.0in
\evensidemargin 0.0in
\textheight     9.0in
\textwidth      6.5in

\title{{\bf An Introduction on Recent Advances in Unsupervised Relation Extraction} \\
\it Term Paper for Recent Advances on Semantics \\ supervised by Prof.Dr.Pinkal}
\author{ {\bf Ehsan Khoddammohammadi}  \\
Faculty of Computational Linguistics \\
Saarland University\\
{\small ehsank@coli.uni-saarland.de}
}
\date{\today}

\begin{document}
\pagestyle{plain}
\pagenumbering{roman}
\maketitle

\pagebreak
\begin{abstract}

The aim of this paper is to review recent and influential methods on
Unsupervised Relation Extraction. In the first part, the definition and a general background
for relation extraction task is provided which identifies important aspects and
difficulties of the task. In the second part, five major previous works are
briefly reviewed. An extra section will be dedicated to a weakly supervised method, distant supervision, which is 
necessary due to its impact on recent advances in the task. And
finally in the last part, the paper will be concluded by comparing the discussed methods.


\end{abstract}

\pagebreak
\tableofcontents
\pagebreak

\cleardoublepage
\pagenumbering{arabic}

\section{Introduction}
\label{ch:intro}

This part provides an overall introduction of the unsupervised relation extraction task, including
definition and various formulations of the task.

\subsection{Definition}
\label{ch:intro}

Relation Extraction is the task of detecting and classifying semantic rela-
tionship between named entities (NE). The goal of this task is to find a triple
of binary relations and their arguments. For instance, we want to induce a
relation like \emph{bornIn} with its arguments which could be for example like this:\emph{ (
bornIn, Richard Stallman, New York )}. Applications of such task is numerous
in natural language processing; Question/Answering, machine translation and
text summarization are systems that benefit from relation extraction.

In this task, we are trying to have a model to find paraphrases which means
that we are interested to have all the similar semantically similar relations under
one umbrella. So it is desired to have all different surface realizations of one
relation like \emph{isGivenBirth, isBorn, isFrom} in a same set, namely \emph{bornIn}.

Here, in this paper, we only limit ourselves to review unsupervised relation discovery methods. 
One should mention that there are other approaches for this task based on the extent of resources that they use.
 The different category of approaches is dependent on if they heavily use annotated corpora (supervised learning) or 
 small amount of training data (weakly supervised learning). Here, in contrast, 
 We just discuss methods that do not use annotated resources for inducing the relations and finding the arguments. 



\section{Major Recent Works on Unsupervised Relation Extraction}
\label{ch:related}

From a classic method, DIRT, to very famous frameworks ,
TextRunner or distant supervision, and more recent works e.g. PATTY,
all are examples of several different family of approaches. These methods could
be categorized from different perspectives (1) amount of annotated data they
need (2) if they can only handle a predefined enumeration of entities and re-
lations or are open to any number of relations (3) organization of semantic
interpretation and (4) underlying family of methods they use.
he content of your proposal. Each topic occupies one section, each
with their own conclusion and future work.
In this section we will review these major works. We will discuss their formulation of the task, their models
 and the features they incorporate, the model constraints and finally we will take a look at how good a model is performing.


\subsection{DIRT}
\label{ch:unsupervised}

\subsection{TextRunner}
\label{ch:unsupervised}

\subsection{USP}
\label{ch:unsupervised}

\subsection{PATTY}
\label{ch:unsupervised}

\subsection{Rel-LDA \& Type-LDA}
\label{ch:unsupervised}





\section{Weakly Supervised on Relation Extraction Task}
\label{ch:related}

\subsection{Distant Supervision}
\label{ch:weakly supervised}

The content of \cite{Huang2012} your proposal. Each topic occupies one section, each
with their own conclusion and future work.

\section{Conclusion}
\label{ch:conclusion}

\subsection{Analysis}
\label{ch:conclusion}

Based on what we have seen in recent works, we can now give a list of vital
attributes that a state-of-the-art model for extracting relations from open text
should be able to carry out. The author will use these facts to suggest a list of possible improvements
in the next section. Modeling all of
these factors in a joint model is the necessary step to push forward the previous
works.

\begin{description}
  
  \item[The number of relations and entities is an unknown parameter.] \hfill \\
  The model can not be confined to a limited set of relations or entities. Being able to extract relations in
  open domain text is the first and (most likely) a trivial attribute of the model. More non-trivial feature
  of the model should be its ability to extract as many relations as there are in text. 
  Giving this freedom
  about the model complexity to have no assumption about the exact number of entities and relations
  is suggested by the applicant to be beneficial 
   and is also supported in the literature.
   %\citep{Yates2007}\citep{Mintz2009} 

  \item[Relations may not be expressed explicitly in text.] \hfill \\
  Relation extraction task is definitely more than finding paraphrases. The model should be able to handle
  long-distance relations among entities as well as hidden semantic indications of a relation.
  
  %\citep{PoonDomingos2009} 

  \item[Relations and entities have their inner organization and types.] \hfill \\
  It is shown by several recent works that relations of relations play a substantial role in identifying
  relations. Relations and entities belong to a hierarchy of types and therefore the constraints they put on each other
  should be learnt as well.
  %\citep{Yao2011}\citep{Alfonesca2012}\citep{NakasholeWeikum2012}

  \item[Using KB is necessary but not enough.] \hfill \\
  There are more relations among entities than what is collected in 
  Knowledge Bases e.g. Freebase. At the same time, it is statistically shown that a supervision
  from such resources strongly contributes to convergance of any model to a better 
  objective configurations.
  %\citep{Mintz2009}\citep{Bordes2011}\citep{Yao2011}

  \item[Relations and entities are sharing information within each other.] \hfill \\
  Relations and entities should be learnt jointly since they share same explanatory factors (hidden variables)
  . Meaning of a entity can be learnt from its relation to other entities and same argument holds among relations.
  Basically, the model should be able to carry multi-task learning. 
  %\citep{Yao2011}\citep{Bordes2011}
    
\end{description}


\subsection{Summaru}
\label{ch:conclusion}


Provide an overview of what you have done and what need to be done.

%\subsection{Plan for completion of the research}

%Table \ref{tab:plan} shows my plan for completion of the research.

%\begin{table}[hc]
%\begin{small}
%\begin{center}
%\begin{tabular}{lll}
%Timeline & Work & Progress\\
%\hline
%          & XXXXXXXXXXXXXXXXXXXXXXXXXXXXXXXXXXXXX & completed\\
%Nov. xxxx & XXXXXXXXXXXXXXXXXXXXXXXXXXX & ongoing\\
%Jan. xxxx & Thesis writting & \\
%Feb. xxxx & Thesis defense & \\
%\end{tabular}
%\end{center}
%\end{small}
%\caption{Plan for completion of my research}
%\label{tab:plan}
%\end{table}



\pagebreak

\begin{footnotesize}
\bibliographystyle{plain}
%\bibliographystyle{agsm}
\bibliography{mendely.bib}
%\printbibliography
\end{footnotesize}

\end{document}


